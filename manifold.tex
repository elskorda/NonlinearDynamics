\section{Πολλαπλότητες-\textlatin{Manifold}}

Πολλαπλότητα είναι ένας τοπολογικός χώρος που είναι τοπικά Ευκλίδειος και  μπορεί να είναι συμπαγής ή μη συμπαγής , συνεκτικός ή μη συνεκτικός. Ένας τοπολογικός χώρος είναι συνεκτικός όταν μπορεί να γραφεί σαν ένωση δύο ξένων συνόλων. Βασικό παράδειγμα μιας πολλαπλότητας είναι ο Ευκλείδειος χώρος και η τρισδιάστατη σφαίρα. Παρακάτω δίνεται ο ορισμός της πολλαπλότητας.
%------------------------------------
\begin{definition}
	Μία  \textlatin{n}-διάστατη πραγματική ομαλή πολλαπλότητα Μ είναι ένας τοπολογικός χώρος που είναι εφοδιασμένος με ένα σύνολο από ανοιχτά σύνολα $U^a$ τέτοια ώστε: 
	\begin{itemize}
		\item για κάθε $p \in M $ υπάρχει κάποιο $U^a$ με $p \in U^a$.
		\item για κάθε $U^a$ υπάρχει ένας αντιστρέψιμος ομοιομορφισμός \footnote{Ομοιομορφισμός ονομάζεται μια απεικόνιση ανάμεσα σε τοπολογικούς χώρους η οποία είναι ένα προς ένα και επί, συνεχής και η αντίστροφη απεικόνιση είναι επίσης συνεχής.} $\phi_a: ~ U^a \rightarrow \mathbb{R}^n $ πάνω σε ένα ανοιχτό υποσύνολο του $\mathbb{R}^n$ τέτοιο ώστε αν $U^a \cap U^b \neq \emptyset $  τότε η απεικόνιση 
			\[ \phi_b \circ \phi_a ^{-1} :\phi_a (U^a \cap U^b) \rightarrow \phi_b(U^a \cap U^b)\]
		είναι ομαλή (απείρως διαφορίσιμη) ως συνάρτηση στον $\mathbb{R}^n$
	 \end{itemize}
\end{definition}
%------------------------------------
Αντίστοιχα θα μπορούσε να ορισθεί και η μιγαδική πολλαπλότητα με την μόνη διαφορά ότι ο πραγματικός χώρος αντικαθίσταται από τον μιγαδικό.\\
Από τον ορισμό της πολλαπλότητας γίνεται κατανοητό ότι κάποιες πολλαπλότητες είναι δύσκολο να διαχωριστούν καθώς ο τοπολογικός χώρος μπορεί να αντιμετωπισθεί ως γεωμετρικό  αντικείμενο και ο ομοιομορφισμός  ουσιαστικά είναι ένα “συνεχές τέντωμα και λύγισμα” του αντικειμένου σε νέο σχήμα. Κατά συνέπεια ένας κύκλος είναι τοπολογικά ίδιος με ένα οποιοδήποτε κλειστό βρόγχο, όσο και αν αυτές οι πολλαπλότητες φαίνονται διαφορετικές. Παρόμοια, η επιφάνεια μιας κούπας του καφέ (με χερούλι)  είναι τοπολογικά ίδια με την επιφάνεια ενός  τόρου. Ένας σημαντικός στόχος για την τοπολογία  είναι να ανακαλύψει ένα τρόπο να διαχωρίζει τις πολλαπλότητες. \\

Οι ομοιομορφισμοί  $\phi_a: ~ U^a \rightarrow \mathbb{R}^n $  ονομάζονται χάρτες συντεταγμένων και παρέχουν στη πολλαπλότητα δομή τοπολογικού χώρου και επιτρέπουν να εκφράζονται τα σημεία μιας μικρής περιοχής σε μια πολλαπλότητα Μ ως συντεταγμένες στον Ευκλίδειο χώρο $\mathbb{R}^n $ (\ref{coorcharts}).

\begin{figure}[H]\centering
	\includegraphics[width =0.4\textwidth]{pics/figure_two11.jpg}
	\caption{Χάρτες συντεταγμένων σε μια πολλαπλότητα.}
	\label{coorcharts}
\end{figure}
\subsection*{Αλλαγή συντεταγμένων }
Είναι δυνατό  εκτός από τους χάρτες που χρησιμοποιούνται στον ορισμό της πολλαπλότητας να συνδεθούν και άλλοι χάρτες  για τους οποίους θα πρέπει να ισχύει ότι $\phi \circ \phi_a^{-1}$ είναι ομαλή στην τομή $\phi_ a (U \cap U^a)$. Επίσης είναι δυνατό να γίνει σύνθεση ενός τοπικού χάρτη συντεταγμένων με ένα διαφορομορφισμό \footnote{διαφορομορφισμός ονομάζεται μια διαφορίσιμη απεικόνιση αν είναι αμφιμονοσήμαντη και η αντίστροφή της είναι επίσης διαφορίσιμη}    που αναφέρεται ως αλλαγή συντεταγμένων. Επειδή η σύνθεση επίσης τοπικός χάρτης συντεταγμένων καθε αντικείμενο πάνω στην Μ ή ιδιότητα πρέπει να είναι ανεξάρτητη από την  αλλαγή συντεταγμένων. Είναι απαραίτητη η προσοχή κατά τον ορισμό αντικειμένων στην περίπτωση που γίνεται με την χρήση κάποιου τύπου σε συγκεκριμένο χάρτη συντεταγμένων γιατι είναι δυνατό οι τύποι να αλλάζουν με την αλλαγή των συντεταγμένων. Σ' αυτή την περίπτωση είναι απαραίτητο να ελέγχεται ο ορισμός αν παραμένει αμετάβλητος κατά την αλλαγή. \\

\subsection*{Συνθήκη μέγιστου βαθμού }
Παρακάτω δίνεται ο ορισμός μιας ομαλής απεικόνισης ανάμεσα σε δύο πολλαπλότητες και ακολούθως ο ορισμός του βαθμού  και του  μέγιστου βαθμού μιας ομαλής απεικόνισης.
\begin{definition}
	Έστω Μ μία πολλαπλότητα διάστασης $m$ και μία πολλαπλότητα διάστασης $n$ με χάρτες $ (U^a,x_a) ~ \text{και } ~ (W^A,y_A ) $ αντίστοιχα. Η  απεικόνιση  $f: ~ M \rightarrow N $ θα λέγεται ομαλή αν για κάθε $ U^a ~\text{και } ~W^A $ τέτοια ώστε  $f(U^a) \cap W^A \neq \emptyset$ η απεικόνιση 
\[ y_a \circ f \circ x_a ^{-1} : x_a(U^a) \rightarrow y_A(W^A) \] είναι ομαλή σαν συνάρτηση $\mathbb{R}^m \rightarrow \mathbb{R}^n$
\end{definition}

\begin{definition}
	Έστω μια ομαλή απεικόνιση  $f: ~ M \rightarrow N $ ανάμεσα σε  Μ μία πολλαπλότητα διάστασης $m$ και μία πολλαπλότητα διάστασης $n$. Βαθμός της απεικόνισης σε κάποιο σημείο $x \in M $ είναι ο βαθμός του Ιακωβιανού πίνακα $\lbr \party{f^j}{x^j}  \rbr ~ \text{στο} ~ x ~ \text{όπου} ~ y=f(x) $ εκφράζεται σε οποιεσδήποτε τοπικές συντεταγμένες κοντά στο $x$. Η απεικόνιση θα είναι μέγιστου βαθμού σε ενα υποσύνολο $S \subset M $ αν για κάθε $x \in S$ ο βαθμός της απεικόνισης είναι όσο μεγαλύτερος γίνεται.
\end{definition}

\begin{theorem}
 	Έστω ότι απεικόνιση  $f: ~ M \rightarrow N $ είναι μέγιστου βαθμού στο $x_0 \in M $, τότε υπάρχουν τοπικές συντεταγμένες $x=(x^1, \ldots x^m)$ κοντά στο $x_0 $ και $y=(y^1, \ldots , y^m) $ κοντά στο $y_0= f(x_0)$. Για τις τοπικές αυτές συντεταγμένες η απεικόνιση παίρνει την απλή μορφή: 
 	\[y=(x^1, \ldots x^m,0,\dots,0) \text{αν} n>m\] ή
 	\[y=(x^1, \ldots x^m) \text{αν} n \leq m\]
\end{theorem}
\subsection*{Υποπολλαπλότητες}
Μια υποπολλαπλότητα είναι ένα υποσύνολο μιας πολλαπλότητας η οποία είναι και αυτή πολλαπλότητα αλλά έχει μικρότερη διάσταση. Για παράδειγμα, ο ισημερινός μιας σφαίρας είναι μια υποπολλαπλότητα. Ο αυστηρός μαθηματικός ορισμός της υπολλαπλότητας δίνεται στη συνέχεία.



\newpage

 
\section{Πολλαπλότητες-\textlatin{Manifold}}

Πολλαπλότητα είναι ένας τοπολογικός χώρος που είναι τοπικά Ευκλίδειος και  μπορεί να είναι συμπαγής ή μη συμπαγής , συνεκτικός ή μη συνεκτικός. Βασικό παράδειγμα μιας πολλαπλότητας είναι ο Ευκλείδειος χώρος και η τρισδιάστατη σφαίρα. Παρακάτω δίνεται ο ορισμός της πολλαπλότητας.

\begin{definition}
Μία  \textlatin{n}-διάστατη πραγματική ομαλή πολλαπλότητα Μ είναι ένας τοπολογικός χώρος που είναι εφοδιασμένος με ένα σύνολο από ανοιχτά σύνολα $U^a$ τέτοια ώστε: 
\begin{itemize}
\item για κάθε $p \in M $ υπάρχει κάποιο $U^a$ με $p \in U^a$.
\item για κάθε $U^a$ υπάρχει ένας αντιστρέψιμος ομοιομορφισμός \footnote{Ομοιομορφισμός ονομάζεται μια απεικόνιση ανάμεσα σε τοπολογικούς χώρους η οποία είναι ένα προς ένα και επί, συνεχής και η αντίστροφη απεικόνιση είναι επίσης συνεχής.} $\phi_a U^a \rightarrow \mathbb{R}^n $ πάνω σε ένα ανοιχτό υποσύνολο του $\mathbb{R}^n$ τέτοιο ώστε αν $U^a \bigcap U^b \neq \emptyset $  τότε η απεικόνιση 
	\[ \phi_b \circ \phi_a ^{-1} :\phi_a (U^a \bigcap U^b) \rightarrow \phi_b(U^a \bigcap U^b)\]
είναι ομαλή (απείρως διαφορίσιμη) ως συνάρτηση στον $\mathbb{R}^n$
 \end{itemize}
\end{definition}

Από τον ορισμό της πολλαπλότητας γίνεται κατανοητό ότι κάποιες πολλαπλότητες είναι δύσκολο να διαχωριστούν καθώς ο τοπολογικός χώρος μπορεί να αντιμετωπισθεί ως γεωμετρικό  αντικείμενο και ο ομοιομορφισμός  ουσιαστικά είναι ένα “συνεχές τέντωμα και λύγισμα” του αντικειμένου σε νέο σχήμα. Κατά συνέπεια ένας κύκλος είναι τοπολογικά ίδιος με ένα οποιοδήποτε κλειστό βρόγχο, όσο και αν αυτές οι πολλαπλότητες φαίνονται διαφορετικές. Παρόμοια, η επιφάνεια μιας κούπας του καφέ (με χερούλι)  είναι τοπολογικά ίδια με την επιφάνεια ενός  τόρου. Ένας σημαντικός στόχος για την τοπολογία  είναι να ανακαλύψει ένα τρόπο να διαχωρίζει τις πολλαπλότητες.

Μια υποπολλαπλότητα είναι ένα υποσύνολο μιας πολλαπλότητας η οποία είναι και αυτή πολλαπλότητα αλλά έχει μικρότερη διάσταση. Για παράδειγμα, ο ισημερινός μιας σφαίρας είναι μια υποπολλαπλότητα.

 
\section{Πολλαπλότητες-\textlatin{Manifold}}

Πολλαπλότητα είναι ένας τοπολογικός χώρος που είναι τοπικά Ευκλίδειος και  μπορεί να είναι συμπαγής ή μη συμπαγής , συνεκτικός ή μη συνεκτικός. Ένας τοπολογικός χώρος είναι συνεκτικός όταν μπορεί να γραφεί σαν ένωση δύο ξένων συνόλων. Βασικό παράδειγμα μιας πολλαπλότητας είναι ο Ευκλείδειος χώρος και η τρισδιάστατη σφαίρα. Παρακάτω δίνεται ο ορισμός της πολλαπλότητας.
%------------------------------------
\begin{definition}
	Μία  \textlatin{n}-διάστατη πραγματική ομαλή πολλαπλότητα Μ είναι ένας τοπολογικός χώρος που είναι εφοδιασμένος με ένα σύνολο από ανοιχτά σύνολα $U^a$ τέτοια ώστε: 
	\begin{itemize}
		\item για κάθε $p \in M $ υπάρχει κάποιο $U^a$ με $p \in U^a$.
		\item για κάθε $U^a$ υπάρχει ένας αντιστρέψιμος ομοιομορφισμός \footnote{Ομοιομορφισμός ονομάζεται μια απεικόνιση ανάμεσα σε τοπολογικούς χώρους η οποία είναι ένα προς ένα και επί, συνεχής και η αντίστροφη απεικόνιση είναι επίσης συνεχής.} $\phi_a: ~ U^a \rightarrow \mathbb{R}^n $ πάνω σε ένα ανοιχτό υποσύνολο του $\mathbb{R}^n$ τέτοιο ώστε αν $U^a \cap U^b \neq \emptyset $  τότε η απεικόνιση 
			\[ \phi_b \circ \phi_a ^{-1} :\phi_a (U^a \cap U^b) \rightarrow \phi_b(U^a \cap U^b)\]
		είναι ομαλή (απείρως διαφορίσιμη) ως συνάρτηση στον $\mathbb{R}^n$
	 \end{itemize}
\end{definition}
%------------------------------------
Αντίστοιχα θα μπορούσε να ορισθεί και η μιγαδική πολλαπλότητα με την μόνη διαφορά ότι ο πραγματικός χώρος αντικαθίσταται από τον μιγαδικό.\\
Από τον ορισμό της πολλαπλότητας γίνεται κατανοητό ότι κάποιες πολλαπλότητες είναι δύσκολο να διαχωριστούν καθώς ο τοπολογικός χώρος μπορεί να αντιμετωπισθεί ως γεωμετρικό  αντικείμενο και ο ομοιομορφισμός  ουσιαστικά είναι ένα “συνεχές τέντωμα και λύγισμα” του αντικειμένου σε νέο σχήμα. Κατά συνέπεια ένας κύκλος είναι τοπολογικά ίδιος με ένα οποιοδήποτε κλειστό βρόγχο, όσο και αν αυτές οι πολλαπλότητες φαίνονται διαφορετικές. Παρόμοια, η επιφάνεια μιας κούπας του καφέ (με χερούλι)  είναι τοπολογικά ίδια με την επιφάνεια ενός  τόρου. Ένας σημαντικός στόχος για την τοπολογία  είναι να ανακαλύψει ένα τρόπο να διαχωρίζει τις πολλαπλότητες. \\

Οι ομοιομορφισμοί  $\phi_a: ~ U^a \rightarrow \mathbb{R}^n $  ονομάζονται χάρτες συντεταγμένων και παρέχουν στη πολλαπλότητα δομή τοπολογικού χώρου και επιτρέπουν να εκφράζονται τα σημεία μιας μικρής περιοχής σε μια πολλαπλότητα Μ ως συντεταγμένες στον Ευκλίδειο χώρο $\mathbb{R}^n $ (\ref{coorcharts}).

\begin{figure}[H]\centering
	\includegraphics[width =0.4\textwidth]{pics/figure_two11.jpg}
	\caption{Χάρτες συντεταγμένων σε μια πολλαπλότητα.}
	\label{coorcharts}
\end{figure}
\subsection*{Αλλαγή συντεταγμένων }
Είναι δυνατό  εκτός από τους χάρτες που χρησιμοποιούνται στον ορισμό της πολλαπλότητας να συνδεθούν και άλλοι χάρτες  για τους οποίους θα πρέπει να ισχύει ότι $\phi \circ \phi_a^{-1}$ είναι ομαλή στην τομή $\phi_ a (U \cap U^a)$. Επίσης είναι δυνατό να γίνει σύνθεση ενός τοπικού χάρτη συντεταγμένων με ένα διαφορομορφισμό \footnote{διαφορομορφισμός ονομάζεται μια διαφορίσιμη απεικόνιση αν είναι αμφιμονοσήμαντη και η αντίστροφή της είναι επίσης διαφορίσιμη}    που αναφέρεται ως αλλαγή συντεταγμένων. Επειδή η σύνθεση επίσης τοπικός χάρτης συντεταγμένων καθε αντικείμενο πάνω στην Μ ή ιδιότητα πρέπει να είναι ανεξάρτητη από την  αλλαγή συντεταγμένων. Είναι απαραίτητη η προσοχή κατά τον ορισμό αντικειμένων στην περίπτωση που γίνεται με την χρήση κάποιου τύπου σε συγκεκριμένο χάρτη συντεταγμένων γιατι είναι δυνατό οι τύποι να αλλάζουν με την αλλαγή των συντεταγμένων. Σ' αυτή την περίπτωση είναι απαραίτητο να ελέγχεται ο ορισμός αν παραμένει αμετάβλητος κατά την αλλαγή. \\

\subsection*{Συνθήκη μέγιστου βαθμού }
Παρακάτω δίνεται ο ορισμός μιας ομαλής απεικόνισης ανάμεσα σε δύο πολλαπλότητες και ακολούθως ο ορισμός του βαθμού  και του  μέγιστου βαθμού μιας ομαλής απεικόνισης.
\begin{definition}
	Έστω Μ μία πολλαπλότητα διάστασης $m$ και μία πολλαπλότητα διάστασης $n$ με χάρτες $ (U^a,x_a) ~ \text{και } ~ (W^A,y_A ) $ αντίστοιχα. Η  απεικόνιση  $f: ~ M \rightarrow N $ θα λέγεται ομαλή αν για κάθε $ U^a ~\text{και } ~W^A $ τέτοια ώστε  $f(U^a) \cap W^A \neq \emptyset$ η απεικόνιση 
\[ y_a \circ f \circ x_a ^{-1} : x_a(U^a) \rightarrow y_A(W^A) \] είναι ομαλή σαν συνάρτηση $\mathbb{R}^m \rightarrow \mathbb{R}^n$
\end{definition}

\begin{definition}
	Έστω μια ομαλή απεικόνιση  $f: ~ M \rightarrow N $ ανάμεσα σε  Μ μία πολλαπλότητα διάστασης $m$ και μία πολλαπλότητα διάστασης $n$. Βαθμός της απεικόνισης σε κάποιο σημείο $x \in M $ είναι ο βαθμός του Ιακωβιανού πίνακα $\lbr \party{f^j}{x^j}  \rbr ~ \text{στο} ~ x ~ \text{όπου} ~ y=f(x) $ εκφράζεται σε οποιεσδήποτε τοπικές συντεταγμένες κοντά στο $x$. Η απεικόνιση θα είναι μέγιστου βαθμού  σε ενα υποσύνολο $S \subset M $ αν για κάθε $x \in S$ ο βαθμός της απεικόνισης είναι όσο μεγαλύτερος γίνεται.
\end{definition}

\begin{theorem}
 	Έστω ότι απεικόνιση  $f: ~ M \rightarrow N $ είναι μέγιστου βαθμού στο $x_0 \in M $, τότε υπάρχουν τοπικές συντεταγμένες $x=(x^1, \ldots x^m)$ κοντά στο $x_0 $ και $y=(y^1, \ldots , y^m) $ κοντά στο $y_0= f(x_0)$. Για τις τοπικές αυτές συντεταγμένες η απεικόνιση παίρνει την απλή μορφή: 
 	\[y=(x^1, \ldots x^m,0,\dots,0) ~\text{αν}~ n>m\] ή
 	\[y=(x^1, \ldots x^m) ~ \text{αν} ~ n \leq m\]
\end{theorem}
\subsection*{Υποπολλαπλότητες}
\begin{definition}
Έστω ομαλή πολλαπλότητα Μ. Μια υποπολλαπλότητα  είναι ένα υποσύνολο $N \subset M $ μαζί με μία ομαλή, ένα προς ένα απεικόνιση $ \chi : \tilde{N} \rightarrow N \subset M$ που ικανοποιεί την συνθήκη μέγιστου βαθμού παντού,  το $\tilde{N}$  είναι παραμετρικός χώρος που αντιστοιχεί σε κάποια άλλη πολλαπλότητα και $N = \chi (\tilde{N}) $ είναι η εικόνα του $\chi$ . Η διάσταση του $N$ είναι ίδια με αυτή του $ \tilde{N}$  αλλά μικρότερη της διάστασης του Μ
\end{definition}

Στην ουσία  υποπολλαπλότητα είναι ένα υποσύνολο μιας πολλαπλότητας η οποία είναι και αυτή πολλαπλότητα αλλά έχει μικρότερη διάσταση. Για παράδειγμα, ο ισημερινός μιας σφαίρας είναι μια υποπολλαπλότητα.

Μια κανονική υποπολλαπλότητα Ν μιας πολλαπλότητας Μ, είναι μία πολλαπλότητα που παραμετροποιείται από την $ \chi:  \tilde{N} \rightarrow M $ με την ιδιότητα ότι για κάθε $x$ στο Ν υπάρχουν αυθαίρετα μικρές ανοιχτές περιοχές $U ~\text{του}~ x $ στο Μ τέτοιες ώστε $\chi^{-1}[U \cap N]$ είναι συνεκτικό ανοιχτό σύνολο του $ \tilde{N} $.

Από το θεώρημα που αναφέρθηκε στη προηγούμενη παράγραφο είναι δυνατό να επιτευχθεί ένας χαρακτηρισμός της κανονικότας μέσα από τις τοπικές συντεταγμένες. 
\begin{lemma}
Μια υποπολλαπλότητα $n$ διαστάσεων $N \subset M $ είναι κανονική αν και μόνο αν για κάθε $x_0 \in N $ υπάρχουν τοπικές συντεταγμένες $ x= (x^1, \ldots , x^m) $ που ορίζονται σε μια γειτονιά $U ~\text{του}~ x_0$  τέτοιες ώστε: 
\[ Ν \cap U = \{ x: x^{n+1}= \cdots =x^m=0 \} \].
\end{lemma}

Για να γίνει κατανοητή η έννοια μια πολλαπλότητας που ορίζεται με πεπλεγμένο τρόπο είναι αναγκαίο να αναφερθεί το πως ορίζεται πεπλεγμένα μια επιφάνεια. Έστω μια επιφάνεια $S$ στον τρισδιάστατο Ευκλείδειο χώρο, αυτή εχει την πεπλεγμένη μορφή 
\[S : =\{ F(x,y,z) =0 \} \]
Αν θεωρηθεί ότι η κλίση $ \nabla F= (F_x , F_y , F_z) $ δεν μηδενίζεται πουθενά στο $S$ τότε λόγω του θεωρήματος πεπλεγμένων συναρτήσεων, σε κάθε σημείο $(x_0 , y_0, z_0) $ του $S$ μπορεί η μια από τις $x, y, z $ να γραφεί ως προς τις άλλες δύο. Με αυτό τον τρόπο και αν $F_z(x_0 , y_0, z_0) \neq 0 $ υπάρχει μια περιοχή $U_a $ του $(x_0 , y_0, z_0)$ τέτοια ώστε στο $U_a $ η $S$  να γράφεται ως $z= f(x,y)$ κάποιας ομαλής συνάρτησης $f$ που ορίζεται σε ένα ανοιχτό υποσύνολο του $\tilde{V_a} \subset \mathbb{R}^2$. Έτσι ορίζεται το σύνολο $\tilde{U_a} S\cap U_a $   με την απεικόνιση $\chi_a (x,y,z) : \tilde{U_a} \rightarrow \tilde{V_a}, \chi_a (x,y,z) =(x,y)  $. Ομοίως αν $F_y(x_0 , y_0, z_0) \neq 0 $ και $y= h(x,z)$ ορίζεται  $\tilde{U_b} S\cap U_b $   με την απεικόνιση $\chi_b (x,y,z) : \tilde{U_b} \rightarrow \tilde{V_b}, \chi_b (x,y,z) =(x,y)  $. Ισχύει ότι 

\[  \chi_b \circ \chi_a ^{-1} =  \chi_b (x,y ,f(x,y)) = (x,f(x,y)) \]

η οποία είναι ομαλή και έχει ομαλή αντίστροφη άρα το $S$ είναι υποπολλαπλότητα του $\mathbb{R^3} $. Τέλος δίνεται το παρακάτω θεώρημα : 
\begin{theorem}
Έστω μια ομαλή πολλαπλότητα διάστασης $m$ και μια ομαλή απεικόνιση $F$ του Μ στον $\mathbb{R^n}, ~ n\leq m$. Αν η $F$ είναι μέγιστου βαθμού  στο υποσύνολο $N=\{ x: F(x)=0 \} $ τότε το Ν είναι μια κανονική υποπολλαπλότητα του Μ  διάστασης $(m-n)$ 
\end{theorem}

\newpage

 
\documentclass[12pt]{article}
\usepackage[greek,english]{babel}
\usepackage[utf8x]{inputenc}
\usepackage{amsmath, amssymb, graphicx, float, stackrel, slashed}
\usepackage{epsfig,caption,geometry,subcaption,hyperref}
\usepackage{xcolor, colortbl}
\usepackage{amsthm}

\definecolor{LRed}{rgb}{1,.8,.8}

\captionsetup{
labelfont=bf,
font=small,
format=hang 
}
\hypersetup{
colorlinks=true,
linkcolor=black,
citecolor=black,
urlcolor=blue!50!black
}
\newcommand{\rbr}{
\ensuremath{\right) }
}

\newcommand{\lbr}{
\ensuremath{\left( }
}

\newcommand{\rsbr}{
\ensuremath{\right] }
}

\newcommand{\lsbr}{
\ensuremath{\left[ }
}

\newcommand{\party}[2]{
\ensuremath{\frac{\partial #1}{\partial #2}}
}
\newcommand{\deriv}[2]{
\ensuremath{\frac{d #1}{d#2}}
}

\newtheorem*{theorem}{Θεώρημα}
\newtheorem*{definition}{Ορισμός}


\begin{document}
\selectlanguage{greek}
\begin{titlepage}
\begin{center}
\large{{\sc Α.Π.Θ} ΣΧΟΛΗ ΘΕΤΙΚΩΝ ΕΠΙΣΤΗΜΩΝ ΤΜ. ΦΥΣΙΚΗΣ}\\[0.5cm]
\vspace{1cm}
\LARGE\textbf{Επίλυση Συστήματος Διαφορικών Εξισώσεων 1ης τάξης με εφαρμογή της θεωρίας ομάδων \textlatin{Lie} }\\[1.0cm] 

\large{Σκορδά Ελένη}\\[0.2cm]

\vspace{1cm}
\small{Εργασία για το μάθημα Μη Γραμμική Δυναμική}\\[0.1cm] 
\small{Διδάσκουσα : Ευθυμία Μελετλίδου Επίκουρος καθηγήτρια }\\[0.2cm]

\begin{figure}[H]\centering
\includegraphics[width =0.3\textwidth]{pics/auth_logo_color.jpg}
\end{figure}
\end{center}
\vfill

\centering{\footnotesize Τμήμα Φυσικής, Α.Π.Θ., \today}
\end{titlepage}

\newpage
\tableofcontents
\newpage
\section{Περίληψη}
\section{Εισαγωγή}
\section{Πολλαπλότητες-\textlatin{Manifold}}
\begin{definition}
Μία  \textlatin{n}-διάστατη πραγματική ομαλή πολλαπλότητα Μ είναι ένας τοπολογικός χώρος που είναι εφοδιασμένος με ένα σύνολο από ανοιχτά σύνολα $U^a$ τέτοια ώστε: 
\begin{itemize}
\item για κάθε $p \in M $ υπάρχει κάποιο $U^a$ με $p \in U^a$.
\item για κάθε $U^a$ υπάρχει ένας αντιστρέψιμος ομοιομορφισμός \footnote{Ομοιομορφισμός ονομάζεται μια απεικόνιση ανάμεσα σε τοπολογικούς χώρους η οποία είναι ένα προς ένα και επί, συνεχής και η αντίστροφη απεικόνιση είναι επίσης συνεχής.} $\phi_a U^a \rightarrow \mathbb{R}^n $ πάνω σε ένα ανοιχτό υποσύνολο του $\mathbb{R}^n$ τέτοιο ώστε αν $U^a \bigcap U^b \neq \emptyset $  τότε η απεικόνιση 
	\[ \phi_b \circ \phi_a ^{-1} :\phi_a (U^a \bigcap U^b) \rightarrow \phi_b(U^a \bigcap U^b)\]
είναι ομαλή (απείρως διαφορίσιμη) ως συνάρτηση στον $\mathbb{R}^n$
 \end{itemize}
\end{definition}
\nocite{topologysame}
\nocite{coordinateCharts}
\nocite{manifold}
\nocite{olver2000applications}
\section{Ομάδες \textlatin{Lie}}
\section{Αλγεβρα \textlatin{Lie} }
%-----------------------------------------------------------------------------------------------------------------------------------------------------
%\input{metavoliatom.tex}
%-----------------------------------------------------------------------------------------------------------------------------------------------------

\newpage
\selectlanguage{english}
\bibliography{Applications_of_Lie_Groups_to_Differenti}
\bibliographystyle{unsrt}


\end{document}

\section{Ομάδες \textlatin{Lie}}

Η ομάδα στα μαθηματικά είναι μια αλγεβρική δομή που αποτελείται από ένα σύνολο σημείων και μια πράξη. Η πράξη συνδυάζει οποιαδήποτε δύο στοιχεία  της ομάδας για να κατασκευάσει ένα τρίτο , ενώ  υπακούει  στην προσεταιριστική ιδιότητα, στην ταυτότητα και στην αντιστρεψιμότητα. 
\begin{definition}
Ομάδα $G$ ονομάζεται ένα σύνολο εφοδιασμένο με μία πράξη (συμβολίζεται με $\cdot$) τέτοια ώστε  για  δύο οποιαδήποτε στοιχεία της ομάδας $g,h $ το $g \cdot h $ είναι επίσης στοιχείο της ομάδας. Η πράξη της ομάδας ικανοποιεί τα εξής : 
\begin{itemize}
\item υπάρχει κάποιο στοιχείο  $e \in G $ τέτοιο ώστε $e \cdot g = g\cdot e = g $ για όλα τα $g \in G$ και το οποίο ονομάζεται μοναδιαίο
\item $g \cdot (h\cdot k ) = (g\cdot h ) \cdot k$
\item $g \cdot g^{-1} = g^{-1} \cdot g = e $ (αντιστρεψιμότητα)
\end {itemize}
\end{definition}
\begin{definition}
Ομάδα \textlatin{Lie} $r$- παραμετρική ονομάζεται μια ομάδα η οποία έχει την δομή μια  ομαλής πολλαπλότητας διάστασης $r$ με τέτοιο τρόπο ώστε τόσο η πράξη της ομάδας
  \[m : G\times G \rightarrow G , ~ ~ m(g,h)=g \cdot h, ~ ~ g,h \in G \] 
  και η αντίστροφη 
  \[i : G \rightarrow G ,~~ i(g)= g^{-1}, ~~ g \in G \] 
  είναι ομαλές απεικονίσεις ανάμεσα σε πολλαπλότητες. 
\end{definition}
Η έννοιά της ομάδας είναι στενά συνδεδεμένη με την έννοια της συμμετρίας. Οι ομάδες \textlatin{Lie} είναι συμμετρικές και συχνά συναντώνται σαν το σύνολο  μετασχηματισμών πάνω σε κάποια πολλαπλότητα. Χαρακτηριστικό παράδειγμα είναι η ομάδα των  στροφών στον τρισδιάστατο χώρο $SO(3)$.

\begin{definition}
Ομοιομορφισμός ανάμεσα σε δυο ομάδες \textlatin{Lie} ονομάζεται η απεικόνιση $ \phi : G \rightarrow H $ ανάμεσα σ' αυτές τις ομάδες που σέβεται τις πράξεις των ομάδων: $ \phi(g \cdot \tilde{g}) = \phi(g) \cdot \phi(\tilde{g}), ~ g , tilde{g} \in G$.
Αν η φ έχει ομαλή αντίστροφη τότε η απεικόνιση ονομάζεται ισομορφισμός.
\end{definition}
\newpage